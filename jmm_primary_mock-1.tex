\documentclass{article}
\usepackage{fontspec}
\usepackage{polyglossia}
\usepackage{enumitem}
\usepackage{graphicx}

% Set main font for English
\setmainlanguage{english}
\setmainfont{Noto Sans}

% Set font for Bengali
\setotherlanguage{bengali}
\newfontfamily\bengalifont{Noto Sans Bengali}

\begin{document}
	
	\title{Math Olympiad (Mock Test)}
	\author{}
	\date{}
	\maketitle
	
	\section*{Category - Primary}
	
	\begin{enumerate}[label=\textbf{\arabic*.}]
		\item 
		\begin{bengali}
			রবির বাড়িতে তার চাচা এসেছে। চাচার আপ্যায়নের জন্যে সে মিষ্টির দোকানে যায় এবং সেখানে গিয়ে, ৪৬ টাকা দিয়ে ৫ কেজি রসগোল্লা এবং ৪ কেজি রসমালাই কিনতে পারে অথবা, ৩৬ টাকা দিয়ে ৪ কেজি রসগোল্লা এবং ৩ কেজি রসমালাই কিনতে পারে। কিন্তু, তার চাচা রসমালাই অনেক পছন্দ করেন এবং রসগোল্লা একদমই পছন্দ করেন না। তাহলে, ৪ কেজি রসগোল্লা ও ৭ কেজি রসমালাই কিনতে রবির কত টাকা লাগবে?
		\end{bengali}
		
		Robi’s uncle has come to Robi’s home. For his uncle’s hospitality, he goes to the sweet shop. He can either buy 5 kg Roshogolla and 4 kg Roshmalai for 46 taka or he can buy 4 kg rosgolla and 4 kg roshmalai for 36 taka. However, his uncle likes Rashmalai a lot and doesn’t like Roshogolla at all. So, how much taka will Robi need to buy 4 kg of Roshogolla and 7 kg of Roshmalai?
		
		\item 
		\begin{bengali}
			$a,$ $b,$ $c,$ $d,$ $e,$ $f$ হলো ৬টি ক্রমিক পূর্ণসংখ্যা, যেন $a$ + $b$ + $c$ + $d$ + $e$ + $f$ =75। তবে, $d$ = ?
		\end{bengali}
		
		a, b, c, d, e, f are 6 consecutive integers, such that a + b + c + d + e + f =75. So, d = ?
		
		
		 
		  \item 
		\begin{bengali}
			 $A=\frac{2024}{2023} \times  \frac{2023}{2022} \times \frac{2022}{2021} \times \dots \times \frac{5}{4}$ এবং $B=\frac{1}{2} \times \frac{2}{3} \times \frac{3}{4} \times \dots \times \frac{2023}{2022}$ হলে, $A+ \frac{1}{B}=?$
		\end{bengali}
		
		If, $A=\frac{2024}{2023} \times \frac{2023}{2022} \times \frac{2022}{2021} \times \dots \times \frac{5}{4}$ and $B=\frac{1}{2} \times \frac{2}{3} \times \frac{3}{4} \times \dots \times \frac{2023}{2022}$ then, $A+ \frac{1}{B}=?$
	
	
	\item
		\begin{bengali}
			 চিত্রে $\triangle$$ABC$ ও $\triangle$$DEF$ দুটি সমবাহু ত্রিভুজ এবং $BC$ ও $DE$ সমান্তরাল। $ABC$ ও $DEF$ এর পরিসীমা যথাক্রমে  12 ও 15। ছায়াকৃত অংশের পরিসীমা কত?
		\end{bengali}
		
		In the figure $\triangle$ABC and $\triangle$DEF are two equilateral triangles and BC and DE are parallel. Ranges of $\triangle$ABC and $\triangle$DEF are 12 and 15 respectively. What is the range of the shaded part?
	\begin{center}
		\includegraphics[width=0.5\textwidth]{jmc_primary_mock-1.png}
	\end{center}

	\item 
	\begin{bengali}
		চিত্রে বহুভুজটির কোণের সমষ্টি কত?
	\end{bengali}
	
	What is the sum of the angles of the polygon in the figure?
	
	\begin{center}
		\includegraphics[width=0.5\textwidth]{polygonfigure.png}
	\end{center}
    \item 
\begin{bengali}
	 জারিফ একটি বাড়ির ৪র্থ ঘরে যেতে চায় এবং সেখানে যাওয়ার জন্য তাকে ৪টি দরজা পার করতে হবে। যেখানে ১ম ঘরে প্রবেশের জন্য ৪টি দরজা, ২য় ঘরে প্রবেশের জন্য ৫টি দরজা, ৩য় ঘরে প্রবেশের জন্য ৩টি দরজা ও ৪র্থ ঘরে প্রবেশের জন্য ২টি দরজা আছে। তাহলে লাবিব কতভাবে ৪র্থ ঘরে যেতে পারবে?
\end{bengali}

Zarif wants to go to the 4th room of a house and has to pass through 4 doors to get there. Where there are 4 doors to enter the 1st room, 5 door to enter the 2nd room, 3 door to enter the 3rd room and 2 doors to enter the 4th room. In how many ways can Zarif go to the 4th room?

\item 
\begin{bengali}
	 1026 থেকে 2025 পর্যন্ত  এমন কতগুলো পূর্ণসংখ্যা আছে যারা 5 দ্বারা বিভাজ্য কিন্তু 4 দ্বারা বিভাজ্য নয়?
\end{bengali}

How many integers from 1026 to 2025 are divisible by 5 but not divisible by 4?

\item 
\begin{bengali}
	 কোচ মুডির ‘চার অধিনায়ক তত্ত্ব’ অনুসারে রংপুর রাইডার্স দলের 25\% সিদ্ধান্ত নেবে সাকিব আল হাসান, 30\% সিদ্ধান্ত নেবে নুরুল হাসান সোহান, 25\% সিদ্ধান্ত নেবে বাবর আজম। অপর অধিনায়ক মোহাম্মদ নবী যদি কোনো এক ম্যাচে 5টি সিদ্ধান্ত নেয় তবে ঐ ম্যাচে কয়টি সিদবধান্ত নেওয়া হয়েছিলো? 
\end{bengali}

      According to coach Moody’s ‘Four Captain Theory’ Shakib Al Hasan will decide 25\% of the Rangpur Riders team, Nurul Hasan Sohan will decide 30\%, Babar Azam will decide 25\%. If the other captain Mohammad Nabi takes 5 decisions in a match, how many decisions were taken in that match?

	\end{enumerate}
	
\end{document}
